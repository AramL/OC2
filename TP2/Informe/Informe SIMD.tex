\documentclass[a4paper]{article}
\usepackage[spanish]{babel}
\usepackage[utf8]{inputenc}
\usepackage{charter}   % tipografia
\usepackage{graphicx}
%\usepackage{makeidx}
\usepackage{paralist} %itemize inline

%\usepackage{float}
%\usepackage{amsmath, amsthm, amssymb}
%\usepackage{amsfonts}
%\usepackage{sectsty}
%\usepackage{charter}
%\usepackage{wrapfig}
%\usepackage{listings}
%\lstset{language=C}


\input{codesnippet}

\usepackage{fancyhdr}

\pagestyle{fancy}

%\renewcommand{\chaptermark}[1]{\markboth{#1}{}}
\renewcommand{\sectionmark}[1]{\markright{\thesection\ - #1}}

\fancyhf{}

\fancyhead[LO]{Sección \rightmark} % \thesection\ 
\fancyfoot[LO]{\small{Leandro Raffo, Maximiliano Fernández Wortman, Uriel Rozenberg.}}
\fancyfoot[RO]{\thepage}

\renewcommand{\headrulewidth}{0.5pt}
\renewcommand{\footrulewidth}{0.5pt}
\setlength{\hoffset}{-0.8in}
\setlength{\textwidth}{16cm}
%\setlength{\hoffset}{-1.1cm}
%\setlength{\textwidth}{16cm}
\setlength{\headsep}{0.5cm}
\setlength{\textheight}{25cm}
\setlength{\voffset}{-0.7in}
\setlength{\headwidth}{\textwidth}
\setlength{\headheight}{13.1pt}

\renewcommand{\baselinestretch}{1.1}  % line spacing




% \setcounter{secnumdepth}{2}
\usepackage{underscore}
\usepackage{caratula}
\usepackage{url}

\begin{document}

\thispagestyle{empty}
\materia{Organización del Computador II}
\submateria{Segundo Cuatrimestre de 2015}
\titulo{Trabajo Práctico II}
\subtitulo{Programacion SIMD}
\integrante{Leandro Raffo}{}{}
\integrante{Maximiliano Fernández Wortman}{}{}
\integrante{Uriel Rozenberg}{}{}

%Pagina de titulo e indice
%\thispagestyle{empty}

\maketitle 

\tableofcontents

\newpage

\section{Introduccion}
En este trabajo práctico realizamos la implementación de dos filtros de imagenes, con tal de ver que tan eficiente puede llegar a ser (o no) SIMD, los filtros son la diferencia de imagenes y el blur gaussiano, los cuales fueron implementados en lenguaje C (gcc y clang) y assembly, haciendo uso de instrucciones vectoriales. Luego comparamos la performance de estas implementaciones sobre diferentes imagenes y usando herramientas probabilísticas y estadísticas.

\section{Implementacion}

\subsection{Diferencia}



\subsection{Blur gaussiano}

\section{Resultados}

\section{Conclusión}

\end{document}