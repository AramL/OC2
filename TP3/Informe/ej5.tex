\section{Ejercicio 5.}

\noindent Completamos el codigo para las interrupciones ISR32, ISR33 y ISR46 que son las interrupciones de reloj, de teclado y la interrupcion 46 respectivamente .


\begin{codesnippet}
\begin{verbatim}
_isr32:
        pushad
        call fin_intr_pic1
        sub esp, 4
        call game_atender_tick
        add esp, 4
        popad  
        iret
		\end{verbatim}
\end{codesnippet}

\noindent Este codigo  guarda el estado de los registros, llama a la funcion \texttt{fin\_intr\_pic1} , alinea la pila y luego llama a \texttt{game\_atender\_tick} , que es una funcion que va atender esta interrupcion.\\
\\
Analogamente la interrupcion para teclado:

\begin{codesnippet}
\begin{verbatim}
    _isr33:
        pushad
        call fin_intr_pic1
        in al, 0x60
        push eax
        call atender_teclado
        pop eax
        popad  
        iret    
    \end{verbatim}
\end{codesnippet}

\noindent Donde \texttt{atender\_teclado} va a ser la funcion que atienda la interrupcion de teclado. Esta funcion recibe por parametro el codigo de caracter que esta siendo presionado, el cual pusheamos a la pila.
\\
\\
La interrupcion 46 solo va a mover al registro eax el valor hexadecimal 0x42.

\begin{codesnippet}
\begin{verbatim}
_isr46:
    pushad
    call fin_intr_pic1
    mov eax,0x42
    popad
    iret    
    \end{verbatim}
\end{codesnippet}

\newpage

\noindent Luego completamos las funciones \texttt{game\_atender\_tick} y \texttt{atender\_teclado}.

\begin{codesnippet}
\begin{verbatim}
void game_atender_tick(perro_t *perro)
{
    mostrar_reloj();
}
\end{verbatim}
\end{codesnippet}

\noindent Esta funcion, asi misma llama a la funcion \texttt{mostrar\_reloj} que esta defininida en screen.c

\begin{codesnippet}
\begin{verbatim}
void mostrar_reloj() {
        if(contador_reloj==5){
           contador_reloj = 0;
        }
        char c = reloj[contador_reloj];
        contador_reloj++;
        screen_pintar_rect(c, C_FG_WHITE, 0, 79, 1, 1);
}
\end{verbatim}
\end{codesnippet}

\noindent Esta funcion usa la variable global \texttt{contador\_reloj} la cual se inicializa en 0, e itera por los caracteres definidos en el array \texttt{reloj}, luego llama a la funcion \texttt{screen\_pintar\_rect} la cual pinta en pantalla el caracter seleccionado, de color blanco en la posicion (0,79), en un cuadrado de 1x1.
\\
\\
La funcion \texttt{atender\_teclado} tambien definida en screen.h, es basicamente un switch el cual dependiendo de que caracter llega por parametro, llama a la funcion \texttt{pintar\_atender\_teclado} con el codigo ascii correspondiente a esa tecla.

\begin{codesnippet}
\begin{verbatim}
void atender_teclado(unsigned char tecla){
   switch (tecla) {
        case KB_q: pintar_atender_teclado('q'); break;
        case KB_a: pintar_atender_teclado('a'); break;
        case KB_k: pintar_atender_teclado('k'); break;
        case KB_z: pintar_atender_teclado('z'); break;
        case KB_x: pintar_atender_teclado('x'); break;
        case KB_c: pintar_atender_teclado('c'); break;
        case KB_b: pintar_atender_teclado('b'); break;
        case KB_n: pintar_atender_teclado('n'); break;
        case KB_m: pintar_atender_teclado('m'); break;
        default:break;
    }
}

\end{verbatim}
\end{codesnippet}
