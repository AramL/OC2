\section{Ejercicio 5.}

\noindent Completamos el codigo para las interrupciones ISR32, ISR33 y isr70 que son las interrupciones de reloj, de teclado y la interrupcion 46 respectivamente .


\begin{codesnippet}
\begin{verbatim}
_isr32:
        pushad
        call fin_intr_pic1
        cmp dword [debug_mode],0
        je .continuar
        cmp dword [debug_view], 1
        je .fin
.continuar:        
        call sched_atender_tick
        str cx
        shl ax, 3
        cmp ax, cx
        je .fin
        mov [sched_tarea_selector], ax
        jmp far [sched_tarea_offset]
        .fin:
        
        popad  
        iret 
		\end{verbatim}
\end{codesnippet}

\noindent Este codigo  guarda el estado de los registros, llama a la funcion \texttt{fin\_intr\_pic1} ,  luego llama a \texttt{sched\_atender\_tick} , que es una funcion que va atender esta interrupcion. Luego salta a la tarea Idle.\\
\\
Analogamente la interrupcion para teclado:

\begin{codesnippet}
\begin{verbatim}
    _isr33:
        pushad
        call fin_intr_pic1
        in al, 0x60
        push eax
        call atender_teclado
        pop eax
        popad  
        iret    
    \end{verbatim}
\end{codesnippet}

\noindent Donde \texttt{atender\_teclado} va a ser la funcion que atienda la interrupcion de teclado. Esta funcion recibe por parametro el codigo de caracter que esta siendo presionado, el cual pusheamos a la pila.
\\
\\
La interrupcion 70 llama a la funcion \texttt{game\_atender\_teclado} enviandoles los parametros por pila.
\begin{codesnippet}
\begin{verbatim}
_isr70:
        push ecx
        push edx
        push ebx
        push ebp
        push esi
        push edi

        push ecx
        push eax
        call game_atender_pedido
        jmp 0x70:0
        add esp, 8

        pop edi
        pop esi
        pop ebp
        pop ebx
        pop edx
        pop ecx
        iret    
    \end{verbatim}
\end{codesnippet}

\newpage

\noindent Luego completamos las funciones \texttt{sched\_atender\_tick} y \texttt{atender\_teclado}.

\begin{codesnippet}
\begin{verbatim}
ushort sched_atender_tick() {    
    game_atender_tick(scheduler.tasks[scheduler.current].perro);
    scheduler.current = sched_proxima_a_ejecutar();
    game_perro_actual = scheduler.tasks[scheduler.current].perro;
    return scheduler.tasks[scheduler.current].gdt_index;
}
\end{verbatim}
\end{codesnippet}

\noindent Dentro de la funcion \texttt{game\_atender\_tick} , se llama a la funcion \texttt{mostrar\_reloj} que esta defininida en screen.c

\begin{codesnippet}
\begin{verbatim}
void mostrar_reloj() {
        if(contador_reloj==5){
           contador_reloj = 0;
        }
        char c = reloj[contador_reloj];
        contador_reloj++;
        screen_pintar_rect(c, C_FG_WHITE, 0, 79, 1, 1);
}
\end{verbatim}
\end{codesnippet}

\noindent Esta funcion usa la variable global \texttt{contador\_reloj} la cual se inicializa en 0, e itera por los caracteres definidos en el array \texttt{reloj}, luego llama a la funcion \texttt{screen\_pintar\_rect} la cual pinta en pantalla el caracter seleccionado, de color blanco en la posicion (0,79), en un cuadrado de 1x1.
\\
\\
La funcion \texttt{atender\_teclado} tambien definida en screen.h, es basicamente un switch el cual dependiendo de que caracter llega por parametro, llama a la funcion \texttt{pintar\_atender\_teclado} con el codigo ascii correspondiente a esa tecla. En caso de presionarse la $Y$, se entra en llama a la rutina de atencion de debug.

\begin{codesnippet}
\begin{verbatim}
void atender_teclado(unsigned char tecla){
   switch (tecla) {
        case KB_q: game_jugador_lanzar_perro(&jugadorA, TIPO_1, jugadorA.x_cucha, jugadorA.y_cucha); break;
        case KB_e: game_jugador_lanzar_perro(&jugadorA, TIPO_2, jugadorA.x_cucha, jugadorA.y_cucha); break;
        case KB_u: game_jugador_lanzar_perro(&jugadorB, TIPO_1, jugadorB.x_cucha, jugadorB.y_cucha); break;
        case KB_o: game_jugador_lanzar_perro(&jugadorB, TIPO_2, jugadorB.x_cucha, jugadorB.y_cucha); break;
        case KB_w: game_jugador_moverse(&jugadorA,0,  -1); break;
        case KB_a: game_jugador_moverse(&jugadorA, -1,  0); break;
        case KB_s: game_jugador_moverse(&jugadorA, 0,  1); break;
        case KB_d: game_jugador_moverse(&jugadorA, 1,  0); break;
        case KB_i: game_jugador_moverse(&jugadorB,  0, -1); break;
        case KB_j: game_jugador_moverse(&jugadorB,  -1, 0); break;
        case KB_k: game_jugador_moverse(&jugadorB,  0, 1); break;
        case KB_l: game_jugador_moverse(&jugadorB,  1, 0); break;
        case KB_z: game_jugador_dar_orden(&jugadorA, 1); break;
        case KB_x: game_jugador_dar_orden(&jugadorA, 2); break;
        case KB_c: game_jugador_dar_orden(&jugadorA, 3); break;
        case KB_b: game_jugador_dar_orden(&jugadorB, 1); break;
        case KB_n: game_jugador_dar_orden(&jugadorB, 2); break;
        case KB_m: game_jugador_dar_orden(&jugadorB, 3); break;
        case KB_y: atender_debug(); break;	
	
    }
}

\end{verbatim}
\end{codesnippet}
