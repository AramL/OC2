\section{Ejercicio 7.}
\subsection{a)}
	Para el funcionamiento del scheduler lo que usamos fue una variable global scheduler que contenia una lista de tareas, que consistian en punteros a perro y el indice de la gdt de la tarea. Y el indice de esa lista correspondiente a la tarea que se esté ejecutando en ese momento.
	
	Para la lista de tareas definimos una array de 17 elementos, donde el elemento 0 era el de la tarea IDLE. Los indices del 1 al 8 corresponden a los perros del jugador A y por ultimo los indices del 9 al 16 corresponden al del jugador B.

	Iniciamos los valores para el indice de la tarea IDLE, y dejamos nulos los valores de indice de gdt y el puntero a perro, luego se va a usar para saber si hay un perro corriendo en ese indice a partir del valor nulo o no del indice de gdt
\begin{codesnippet}
\begin{verbatim}
    scheduler.tasks[0].gdt_index = GDT_TSS_IDLE;
    scheduler.tasks[0].perro = NULL;
    for (i = 1; i < MAX_CANT_TAREAS_VIVAS + 1; i++) {
        scheduler.tasks[i].gdt_index = NULL;
        scheduler.tasks[i].perro = NULL;
    }
    scheduler.current = 0;
\end{verbatim}
\end{codesnippet}

\subsection{b)}

Para la función proxima a ejecutar definimos una funcion auxiliar que nos daba el proximo perro (indice de la tarea) del jugador siguiente y de no haberlo buscabamos en los perros del jugador actual y en caso de que los dos no tuviesen perros corriendo defaulteabamos a la tarea idle. El código es el siguiente.

\begin{codesnippet}
\begin{verbatim}
uint sched_proxima_a_ejecutar() {
    uint prox_jugador = (scheduler.current <= 8);
    uint prox_task = sched_proximo_perro_jugador(prox_jugador);

    if (!prox_task) {
        prox_jugador = !prox_jugador;
        prox_task = sched_proximo_perro_jugador(prox_jugador);
        if (!prox_task)
            return 0;
    }
    ultimo_index[prox_jugador] = prox_task;
    return prox_task;
}
\end{verbatim}
\end{codesnippet}

y la función sched\_proximo\_perro\_jugador es 

\begin{codesnippet}
\begin{verbatim}
uint sched_proximo_perro_jugador(uint jugador) {
    uint index = ultimo_index[jugador];
    do {
        index++;
        if (index > (8 << jugador))
            index -= 8;
        if (scheduler.tasks[index].gdt_index != NULL)
            return index;
    } while (index != ultimo_index[jugador]);
    return 0;
}
\end{verbatim}
\end{codesnippet}

Donde básicamente loopeamos desde la ultima tarea ejecutada (sin considerarla) en adelante, si no encontramos tareas en ese intervalo, restamos 8 y empezamos desde el principio hasta la ultima tarea ejecutada inclusive, en caso de no encontrar una tarea 
libre devolvemos 0. notar que 8 $<<$ jugador da el valor correcto en el que restamos 8, si el jugador es el A, es decir el 0, restamos 8 cuando index sea igual a 9, y si es el jugadorB es decir el 1, el shift hace que 8 sea 16 y restamos cuando index es 17. 



\subsection{c) y e)}
 Se setea la tarea que se esta ejecutando como la nueva tarea a ser ejecutada. 
\begin{codesnippet}
\begin{verbatim}
	scheduler.current = sched_proxima_a_ejecutar();
    game_perro_actual = scheduler.tasks[scheduler.current].perro;
    \end{verbatim}
\end{codesnippet}

En la interrupcion de reloj se agrego el siguiente codigo, tomando la proxima tarea a ejecutar y haciendo la conmutacion de tareas
  \begin{codesnippet}
\begin{verbatim}
        call sched_atender_tick
        ;xor ecx, ecx
        str cx
        shl ax, 3
        cmp ax, cx
        je .fin
        mov [sched_tarea_selector], ax
        jmp far [sched_tarea_offset]
            \end{verbatim}
\end{codesnippet}

\subsection{d)}
Se cambio un poco el codigo de la interrupcion para poder devolver EAX como resultado,
Asi como tambien el jmp a la tarea idle, usando 0x70 como selector, dado que 0x70 = 1110000b, si tomamos la parte del indice nos queda 1110 que equivale a 14 en decimal, el indice de la gdt de la tarea IDLE
  \begin{codesnippet}
\begin{verbatim}
  push ecx
        push edx
        push ebx
        push ebp
        push esi
        push edi

        push ecx
        push eax
        call game_atender_pedido
        jmp 0x70:0
        add esp, 8

        pop edi
        pop esi
        pop ebp
        pop ebx
        pop edx
        pop ecx
        iret    
   \end{verbatim}
\end{codesnippet}

Para mover el perro  habia que chequear que la nueva posicion fuese valida
  \begin{codesnippet}
\begin{verbatim}

 int x, y;
    game_dir2xy(dir, &x, &y);
    int nuevo_x = perro->x + x;
    int nuevo_y = perro->y + y;

    if(!game_es_posicion_valida(nuevo_x, nuevo_y))
        return 0;

    if(game_perro_en_posicion_j(perro->jugador, nuevo_x, nuevo_y) != NULL)
        return 0;
   \end{verbatim}
\end{codesnippet}
 
Se tiene que hacer el proceso de mover el perro en si, mapeamos la nueva direccion y copiamos entero al codigo del perro y su pila
, luego cambiamos en paginacion la traduccion de la direcion 0x401000 a la nueva posicion escrita. y por ultimo mapear la posicion del perro solo como lectura.
  \begin{codesnippet}
\begin{verbatim}
    uint dir_fisica = mmu_xy2fisica(nuevo_x, nuevo_y);
    uint dir_virtual = mmu_xy2virtual(nuevo_x, nuevo_y);

    mmu_mapear_pagina(0x7FFF000, rcr3(), dir_fisica, 0x3);
    mmu_copiar_pagina(0x401000, 0x7FFF000);

    mmu_mapear_pagina(0x401000, rcr3(), dir_fisica, 0x7);
    mmu_mapear_pagina(dir_virtual, rcr3(), dir_fisica, 0x5);

    mmu_unmapear_pagina(0x7FFF000, rcr3());
   \end{verbatim}
\end{codesnippet}
 
 Refrescamos en la pantalla la nueva posicion del perro, y nos fijamos si esta en la cucha
  \begin{codesnippet}
\begin{verbatim}
    screen_borrar_perro(perro);
    perro->x = nuevo_x;
    perro->y = nuevo_y;
    screen_pintar_perro(perro);
     game_perro_ver_si_en_cucha(perro);
    return 1;
       \end{verbatim}
\end{codesnippet}
   
En caso de estarlo, anota los puntos y luego termina la tarea del perro
  \begin{codesnippet}
\begin{verbatim}
   void game_perro_ver_si_en_cucha(perro_t *perro)
{
    if (perro->x != perro->jugador->x_cucha || perro->y != perro->jugador->y_cucha){
        return;
    }

    if (perro->huesos == 0){
        return;
    }

    while(perro->huesos > 0){
        game_jugador_anotar_punto(perro->jugador);
        perro->huesos--;

    }
  
    game_perro_termino(perro);
}
       \end{verbatim}
\end{codesnippet}
\begin{codesnippet}
\begin{verbatim}
void game_jugador_anotar_punto(jugador_t *j) {
    ultimo_cambio = MAX_SIN_CAMBIOS;

    j->puntos++;

    screen_pintar_puntajes();

    if (j->puntos == 999)
        screen_stop_game_show_winner(j);
}
 \end{verbatim}
\end{codesnippet}
\begin{codesnippet}
\begin{verbatim}
void game_perro_termino(perro_t *perro){
    sched_remover_tarea(perro->id);
}
 \end{verbatim}
\end{codesnippet}

Para sacar al perro que ya termino, buscamos el indice dentro de la lista de tareas a partir del gdt\_index recibido,
borra al perro de la pantalla, libera al perro y saca al perro del scheduler
\begin{codesnippet}
\begin{verbatim}
void sched_remover_tarea(unsigned short gdt_index) {

    uint i = 1;
    while (scheduler.tasks[i].gdt_index != gdt_index)
        i++;
    perro_t *p = scheduler.tasks[i].perro;
    screen_borrar_perro(p);
    p->libre = TRUE;
    screen_pintar_reloj_perro(p);
    scheduler.tasks[i].gdt_index = NULL;
    if (i == scheduler.current)
        scheduler.current = NULL;
}
 \end{verbatim}
\end{codesnippet}

\subsection*{g)}

\noindent Para este punto, se agrego las variables $debugMode$ y $debugView$. La primera indica si se entro en modo debug, y la segunda indica si en este momento se esta mostrando informacion en la pantalla. \\

\noindent Luego se cambiaron tanto la rutina de atencion de interrupciones de teclado, para agregar la tecla Y, la cual setea estas variables, como el macro para la atencion de excepciones. Este se modifico para que en caso de encontrarse en modo debug, se cree una copia de la pantalla, y luego se proceda a mostrar la informacion de la excepcion requerida por el enunciado. \\

\noindent Se utilizo un arreglo para mantener una copia de la pantalla anterior.

\begin{codesnippet}
\begin{verbatim}
    short pantalla[80 * 50];
    .....
    .....
void game_guardar_pantalla() {
    short *src = (short *)0xB8000;
    int i;
    for(i = 0; i < 80 * 50; i++){
        pantalla[i] = src[i];
     }
}
\end{verbatim}
\end{codesnippet}

\noindent La rutina de atencion de interrupciones del reloj, se modifico para que en caso de encontrarse en debug, no se ejecute el salto de tarea sino que se espera hasta que se desactive el mismo.
\begin{codesnippet}
\begin{verbatim}
        cmp dword [debug_mode],0
        je .continuar
        cmp dword [debug_view], 1
		je .fin
.continuar:
        ....
        rutina de atencion reloj
        ....
.fin:
        popad  
        iret
\end{verbatim}
\end{codesnippet}

\noindent Para escribir los estados de los flags, como de los registros, los mismos se pushean a la pila y se pasan por parametro para luego imprimirlos por pantalla.

\begin{codesnippet}
\begin{verbatim}
    mov eax, [esp]
    pushf 									
    push eax
    call game_imprimir_info_debug
\end{verbatim}
\end{codesnippet}
