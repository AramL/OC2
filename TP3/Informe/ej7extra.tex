\subsection{b)}

Para la función proxima a ejecutar definimos una funcion auxiliar que nos daba el proximo perro (indice de la tarea) del jugador siguiente y de no haberlo buscabamos en los perros del jugador actual y en caso de que los dos no tuviesen perros corriendo defaulteabamos a la tarea idle. El código es el siguiente.

\begin{codesnippet}
\begin{verbatim}
uint sched_proxima_a_ejecutar() {
    uint prox_jugador = (scheduler.current <= 8);
    uint prox_task = sched_proximo_perro_jugador(prox_jugador);

    if (!prox_task) {
        prox_jugador = !prox_jugador;
        prox_task = sched_proximo_perro_jugador(prox_jugador);
        if (!prox_task)
            return 0;
    }
    ultimo_index[prox_jugador] = prox_task;
    return prox_task;
}
\end{verbatim}
\end{codesnippet}

y la función sched_proximo_perro_jugador es 

\begin{codesnippet}
\begin{verbatim}
uint sched_proximo_perro_jugador(uint jugador) {
    uint index = ultimo_index[jugador];
    do {
        index++;
        if (index > (8 << jugador))
            index -= 8;
        if (scheduler.tasks[index].gdt_index != NULL)
            return index;
    } while (index != ultimo_index[jugador]);
    return 0;
}
\end{verbatim}
\end{codesnippet}

Donde básicamente loopeamos desde la ultima tarea ejecutada (sin considerarla) en adelante, si no encontramos tareas en ese intervalo, restamos 8 y empezamos desde el principio hasta la ultima tarea ejecutada inclusive, en caso de no encontrar una tarea 
libre devolvemos 0. notar que 8 << jugador da el valor correcto en el que restamos 8, si el jugador es el A, es deicr el 0, restamos 8 cuando index sea igual a 9, y si es el jugadorB es decir el 1, el shift hace que 8 sea 16 y restamos cuando index es 17. 